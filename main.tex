% !TeX root = main.tex

\input{preamble.tex}
\begin{document}

\pagestyle{fancy}
\renewcommand{\footrulewidth}{0pt}
\fancyhf{}
\lfoot{\textbf{MATHEUS A. R. M. HORÁCIO | 231107376}}
%\rfoot{\textbf{MATRÍCULA: 17/0110923 }}
\fancyfoot[RO]{\hspace*{2cm} \textbf{Página \thepage \ de \pageref*{LastPage}}}
  \renewcommand\footrule{%

 \color{BlueViolet}\noindent\makebox[\linewidth]{\rule{\paperwidth}{1pt}}
}



\newcommand{\HRule}{\rule{\linewidth}{0.5mm}} % Defines a new command for the horizontal lines, change thickness here
\begin{center} % Center everything on the page
 
%----------------------------------------------------------------------------------------
%	HEADING SECTIONS
%----------------------------------------------------------------------------------------



\PlaceText{15mm}{27mm}{ \color{gal}\noindent\makebox[\linewidth]{\rule{2\paperwidth}{1.5pt}}}

\PlaceText{11mm}{42mm}{\huge\bfseries \color{gal} Ementa do exame de qualificação em Análise}

\PlaceText{61mm}{55mm}{\huge \bfseries \color{gal} (Segunda Área)}


\PlaceText{15mm}{62mm}{ \color{gal}\noindent\makebox[\linewidth]{\rule{2\paperwidth}{1.5pt}}}


\PlaceText{7mm}{20mm}{ \includegraphics[scale=0.7]{unb.eps}}


\PlaceText{131mm}{17mm}{\bfseries \color{gal} Aluno: Matheus A. R. M. Horácio }

\PlaceText{131mm}{23mm}{\bfseries \color{gal} Matrícula: 231107376}


\end{center}
\vspace{0.7cm}

\begin{itemize}
  \item Caracterizações de operadores lineares contínuos
  \item Teorema de Banach-Steinhaus
  \item Teorema da Aplicação Aberta
  \item Teorema do Gráfico Fechado
  \item \underline{\textbf{O teorema de Hahn-Banach}}
  \begin{itemize}
    \item Forma analítica do Teorema de Hahn-Banach e corolários
    \item Primeira e segunda forma geométrica do teorema de Hahn-Banach
  \end{itemize}
  \item \underline{\textbf{Topologias fraca e $\text{fraca}^{\star}$}}
  \begin{itemize}
    \item Conceito de topologia fraca e $\text{fraca}^{\star}$
    \item Compacidade fraca e reflexividade
    \item Espaços uniformemente convexos
  \end{itemize}
  \item \underline{\textbf{Espaços de Hilbert}}
  \begin{itemize}
    \item Espaços com produto interno
    \item Ortogonalidade e projeções
    \item Conjuntos ortonormais em espaços de Hilbert
    \item Teorema da Representação de Riesz
    \item Teorema de Lax-Milgram
  \end{itemize}
  \item \underline{\textbf{ Teoria Espectral de Operadores Compactos e Autoadjuntos}}
  \begin{itemize}
    \item Espectro de um operador contínuo
    \item Operadores compactos
    \item Espectro de operadores compactos
    \item Operadores autoadjuntos em espaços de Hilbert
    \item Espectro de operadores autoadjuntos
  \end{itemize}
  \item \underline{\textbf{ Funções harmônicas}}
  \begin{itemize}
    \item Caracterização de funções harmônicas pela propriedade da média
    \item Princípios do máximo para funções harmônicas
  \end{itemize}
  \newpage
  \item \underline{\textbf{ O problema de Poisson}}
  \begin{itemize}
    \item A solução fundamental e o Potencial Newtoniano
    \item A solução do problema de Perron
    \item A função de Green
  \end{itemize}
  \item \underline{\textbf{ Operadores lineares de segunda ordem}}
  \begin{itemize}
    \item Princípios do máximo para operadores lineares de segunda ordem
    \item Espaços de Hölder, imersões contínuas e compactas
    \item O Teorema de Existência de Schauder
  \end{itemize}
  
  \item \underline{\textbf{ Espaços de Sobolev}}
  \begin{itemize}
    \item Derivada fraca
    \item Aproximações por funções suaves
    \item Teorema da Extensão e Teorema do Traço
    \item Imersões contínuas ou compactas de $W^{k,p}$
  \end{itemize}
  \item \underline{\textbf{Soluções fracas para equações lineares de segunda ordem}}
  \begin{itemize}
    \item Existência de soluções
    \item O teorema de Lax-Milgram e a existência de soluções fracas
    \item A Alternativa de Fredholm e a existência de soluções fracas
  \end{itemize}
  \item \underline{\textbf{Autovalores de operadores elípticos}}
  \begin{itemize}
    \item O espectro de $-\Delta$
    \item Regularidade de soluções
  \end{itemize}
\end{itemize}



\section*{Bibliografia}
\begin{itemize}
  % \item \label{kreyz} Kreyszig, E. C., \& Gariepy, R. (2015). \textit{Measure Theory and Fine Properties of Functions}, Revised edition.
  \item \label{kreyz} Kreyszig, E. \textit{Introductory Functional Analysis with Applications}.
  \item \label{brez} Brezis, H. \textit{Functional Analysis, Sobolev Spaces and Partial Differential Equations}.
  \item \label{evans} Evans, L. C. \textit{Partial Differential Equations}.
\end{itemize}


\end{document}
